\chapter{Units Overview, Commands and Events Description} \label{sec:units-overview}

This chapter describes all functional blocks (units) implemented in GEX, version 1.0. The term ``unit'' will be used here to refer to both unit types (drivers) or their instances where the distinction is not important.

Each unit's description will be accompanied by a corresponding snippet from the configuration file, and a list of supported commands and events. The commands and events described here form the payload of TinyFrame messages 0x10 (Unit Request) and 0x11 (Unit Report), as described in section \ref{sec:unit_requests_reports}. 

The number in the first column of the command (or event) tables, marked as ``Code'', is the command number (or report type) used in the payload to identify how the message data should be treated. When the request or response payload is empty, it is omitted from the table. The same applies to commands with no response, in which case adding 0x80 to the command number triggers a SUCCESS response after the command is finished.

\subsection{Unit Naming}

Unit types are named in uppercase (e.g. SPI, 1WIRE, NPX) in the INI file and in the list of units. Unit instances can be named in any way the user desires; using lowercase makes it easier to distinguish them from unit types. It is advisable to use descriptive names, e.g. not ``pin1'' but rather ``button''.

\subsection{Packed Pin Access} \label{sec:packedpins}

Several units facilitate an access to a group of GPIO pins, such as the digital input and output units, or the SPI unit's slave select pins. The STM32 microcontroller's ports have 16 pins each, most of which can be configured to one of several alternate functions (e.g. SPI, PWM outputs, ADC input). As a consequence, it is common to be left with a discontiguous group of pins after assigning all the alternate functions needed by an application.

\begin{figure}[h]
	\centering
	\includegraphics[scale=1] {img/pin-packing.pdf}
	\caption{\label{fig:pin-packing}Pin packing}
\end{figure}

For instance, we could only have the pins 0, 1, 12--15 available on a \gls{GPIO} port. GEX provides a helpful abstraction to bridge the gaps in the port: The selected pins are packed together and represented, in commands and events, as a block of six pins (0x3F) instead of their original positions in the register (0xF003). This scheme is shown in figure \ref{fig:pin-packing}. The translation is done in the unit driver and works transparently, as if the block of pins had no gaps---all the referenced pins are updated simultaneously without glitches. Where pin numbers are used, the order in the packed word should be provided---in our example, that would be 0--5, counting from the least significant bit.

% here are the unit sections, all following a common pattern
\section{DO: Digital Output}

The digital output unit provides a write access to one or more pins of a GPIO port. This unit additionally supports pulse generation on any of its pins. This is implemented in software with the timing derived from the system timebase, as the hardware timer outputs, otherwise used for PWM or pulse generation, are available only on several dedicated pins. The timing code is optimized to reduce jitter. \todo{Measure jitter and add it here}

\subsection{DO Configuration}

\begin{inicode}
[DO:out@1]
# Port name
port=A
# Pins (comma separated, supports ranges)
pins=0
# Initially high pins
initial=
# Open-drain pins
open-drain=
\end{inicode}

\subsection{DO Events}

This unit generates no events.

\subsection{DO Commands}

\begin{tabularx}{\textwidth}{p{\fldwcode}lXp{\fldwpld}}
	\toprule
	\textbf{Code} & \textbf{Name} & \textbf{Function} & \textbf{Payload}  \\	
	\midrule	
	
	0x00 & WRITE & Write to all pins 
	& \makecell[tl]{
		\fldreq
		\fld{u16} new value
	} \\
	
	0x01 & SET & Set selected pins to 1 
	& \makecell[tl]{
		\fldreq
		\fld{u16} pins to set
	} \\
	
	0x02 & CLEAR & Set selected pins to 0 
	& \makecell[tl]{
		\fldreq
		\fld{u16} pins to clear
	} \\

	0x03 & TOGGLE & Toggle selected pins 
	& \makecell[tl]{
		\fldreq
		\fld{u16} pins to toggle
	} \\

	0x04 & PULSE & Generate a pulse on the selected pins. The $\mu$s scale may be used only for 0--999\,$\mu$s.
	& \makecell[tl]{
		\fldreq
		\fld{u16} pins to pulse \\
		\fld{u8} active level (0, 1) \\
		\fld{u8} scale: 0-ms, 1-$\mu$s \\
		\fld{u16} duration
	} \\
	\bottomrule
\end{tabularx}

\section{Digital Input Unit}

The digital input unit is the input counterpart of the digital output unit. 

In addition to reading the immediate digital levels of the selected pins, this unit can generate asynchronous events on a pin change. The state of the entire input port, together with a microsecond timestamp (as is the case for all asynchronous events), is reported to the host either on a rising, falling, or any pin change. 

The pin change event can be configured independently for each pin. In order to receive a pin change event, it must be armed first; The pin can be armed for a single event, or it may be re-armed automatically with a hold-off time. It's further possible to automatically arm selected pin triggers on start-up.

\section{SIPO (Shift Register) Unit}

The shift registers driver unit is designed for the loading of data into \textit{serial-in, parallel-out} (SIPO) shift registers, such as 74HC4094 or 74HC595. Those are commonly used to control segmented LED displays, LED matrices etc.

This unit handles both the \textit{Shift} and \textit{Store} signals and is capable of loading multiple shift registers simultaneously, reducing visible glitches in the display. It's also possible to set the data lines to arbitrary level(s) before sending the Store pulse, which can be latched and used for some additional feature of the LED display, such as brightness control.


\subsection{SIPO Configuration}

\begin{inicode}
[SIPO:display@9]
# Shift pin & its active edge (1-rising,0-falling)
shift-pin=A1
shift-pol=1
# Store pin & its active edge
store-pin=A0
store-pol=1
# Clear pin & its active level
clear-pin=A2
clear-pol=0
# Data port and pins
data-port=A
data-pins=3
\end{inicode}

\subsection{SIPO Commands}

\begin{cmdlist}
	0 & \cname{WRITE}
	Load the shift registers and leave the data outputs in the "trailing data" state before sending the Store pulse.
	& 
	\begin{cmdreq}
		\cfield{u16} trailing data
		\item For each output (same size)
		\begin{pldlist}
			\cfield{u8[]} data to load
		\end{pldlist}
	\end{cmdreq} 
	\\

	1 & \cname{DIRECT\_DATA}
	Directly write to the data pins (same like the DO unit's WRITE command)
	&
    \begin{cmdreq}
		\cfield{u16} values to write
	\end{cmdreq} \\

	2 & 
	\cname{DIRECT\_CLEAR} 
	Pulse the Clear pin, erasing the registers' data & \\
	
	3 & 
	\cname{DIRECT\_SHIFT}
	Pulse the Shift pin & \\
	
	4 & 
	\cname{DIRECT\_STORE} 
	Pulse the Store pin & \\
\end{cmdlist}




\section{NeoPixel Unit}

The NeoPixel unit implements the protocol needed to control a digital LED strip with WS2812, WS2811, or compatible LED driver chips. The protocol timing is implemented in software, therefore it is available on any GPIO pin of the module.

The color data can be loaded in five different format: as packed bytes, or as the little-endian or big-endian encoding of colors in the 32-bit format 0x00RRGGBB or 0x00BBGGRR. This data format is convenient when the colors are already represented by an array of 32-bit integers.

\subsection{NeoPixel Configuration}

\begin{inicode}
[NPX:neo@3]
# Data pin
pin=A0
# Number of pixels
pixels=32
\end{inicode}

\subsection{NeoPixel Commands}

\begin{tabularx}{\textwidth}{p{\fldwcode}Xp{\fldwpld}}
	\toprule
	\textbf{Code} & \textbf{Function} & \textbf{Payload}  \\	
	\midrule	
	
	0 & \flname{CLEAR}
	Switch all LEDs off (sets them to black) & \\
	
	1 & \flname{LOAD}	
	Load a sequence of R,G,B bytes
	& \makecell[tl]{
		\fldreq
		\tabitem For each LED: \\
		~~\fldo{u8} red \\
		~~\fldo{u8} green \\
		~~\fldo{u8} blue \\
	} \\

	4 & \flname{LOAD\_U32\_ZRGB}
	Load 32-bit big-endian 0xRRGGBB (0,R,G,B)
	& \makecell[tl]{
		\fldreq
		\fld{u32[]} color data BE
	} \\

	5 & \flname{LOAD\_U32\_ZBGR}
	Load 32-bit big-endian 0xBBGGRR (0,B,G,R)
	& \makecell[tl]{
		\fldreq
		\fld{u32[]} color data BE
	} \\

	6 & \flname{LOAD\_U32\_RGBZ}
	Load 32-bit little-endian 0xBBGGRR (R,G,B,0)
	& \makecell[tl]{
		\fldreq
		\fld{u32[]} color data LE
	} \\

	7 & \flname{LOAD\_U32\_BGRZ}
	Load 32-bit little-endian 0xRRGGBB (B,G,R,0)
	& \makecell[tl]{
		\fldreq
		\fld{u32[]} color data LE
	} \\

	10 & \flname{GET\_LEN}
	Get number of LEDs in the strip
	& \makecell[tl]{
		\fldresp
		\fld{u16} number of LEDs
	} \\
	\bottomrule
\end{tabularx}




\section{SPI Unit}

The SPI unit provides access to one of the microcontroller's SPI peripherals. It can be configured to use any of the different speeds, clock polarity and phase settings available in its control registers. 

The unit handles up to 16 slave select (NSS) signals and supports message multi-cast (addressing more than one slaves at once). Protection resistors should be used if a multi-cast transaction is issued with MISO connected.

The QUERY command of this unit, illustrated by figure \ref{fig:spi_query}, is flexible enough to support all types of SPI transactions: read-only, write-only, and read-write with different request and response lengths. The slave select pin is held low during the entire transaction.

\begin{figure}[h]
	\centering
	\includegraphics[scale=1.1] {img/spi-query.pdf}
	\caption{\label{fig:spi_query}SPI transaction using the QUERY command}
\end{figure}

\subsection{SPI Configuration}

\begin{inicode}
[SPI:spi@5]
# Peripheral number (SPIx)
device=1
# Pin mappings (SCK,MISO,MOSI)
#  SPI1: (0) A5,A6,A7     (1) B3,B4,B5
#  SPI2: (0) B13,B14,B15
remap=0
# Prescaller: 2,4,8,...,256
prescaller=64
# Clock polarity: 0,1 (clock idle level)
cpol=0
# Clock phase: 0,1 (active edge, 0-first, 1-second)
cpha=0
# Transmit only, disable MISO
tx-only=N
# Bit order (LSB or MSB first)
first-bit=MSB
# SS port name
port=A
# SS pins (comma separated, supports ranges)
pins=0
\end{inicode}

\subsection{SPI Events}

This unit generates no events.

\subsection{SPI Commands}

\begin{tabularx}{\textwidth}{p{\fldwcode}lXp{\fldwpld}}
	\toprule
	\textbf{Code} & \textbf{Name} & \textbf{Function} & \textbf{Payload}  \\	
	\midrule	
	
	0x00 & QUERY & Exchange bytes with a slave device
	& \makecell[tl]{
		\fldreq
		\fld{u8} slave number 0--16 \\
		\fld{u16} response padding \\
		\fld{u16} response length \\
		\fld{u8[]} bytes to write \\
		\fldresp
		\fld{u8[]} received bytes \\		
	} \\
	
	0x01 & MULTICAST & Send a message to multiple slaves at once. The address is a bit map (e.g. 0x8002 = slaves 1 and 15).
	& \makecell[tl]{
		\fldreq
		\fld{u16} addressed slaves \\
		\fld{u8[]} bytes to write
	} \\
	\bottomrule
\end{tabularx}
















\section{\texorpdfstring{\IIC}{I2C} Unit}

The \gls{I2C} unit provides access to one of the microcontroller's \gls{I2C} peripherals. More on the \IIC bus can be found in \cref{sec:theory-i2c}.

The unit can be configured to use either of the three standard speeds (Standard, Fast and Fast+) and supports both 10-bit and 7-bit addressing. 10-bit addresses can be used in commands by setting their highest bit (0x8000), as a flag to the unit; the 7 or 10 least significant bits will be used as the actual address.

\subsection{\texorpdfstring{\IIC}{I2C} Configuration}

\begin{inicode}
[I2C:i2c@4]
# Peripheral number (I2Cx)
device=1
# Pin mappings (SCL,SDA)
#  I2C1: (0) B6,B7    (1) B8,B9
#  I2C2: (0) B10,B11  (1) B13,B14
remap=0

# Speed: 1-Standard, 2-Fast, 3-Fast+
speed=1
# Analog noise filter enable (Y,N)
analog-filter=Y
# Digital noise filter bandwidth (0-15)
digital-filter=0
\end{inicode}

\subsection{\texorpdfstring{\IIC}{I2C} Commands}

\begin{cmdlist}

	0 & \cname{WRITE}
	Perform a raw write transaction
	& \begin{cmdreq}
		\cfield{u16} slave address
		\cfield{u8[]} bytes to write
	\end{cmdreq} \\

	1 & \cname{READ}
	Perform a raw read transaction.
	& \begin{cmdreq}
		\cfield{u16} slave address
		\cfield{u16} number of read bytes
    \end{cmdreq}
	\cjoin
    \begin{cmdresp}
		\cfield{u8[]} received bytes
    \end{cmdresp} \\

	2 & \cname{WRITE\_REG}
	Write to a slave register. Sends the register number and the data in the same transaction. Multiple registers can be written at once if the slave supports auto-increment.
	& \begin{cmdreq}
		\cfield{u16} slave address
		\cfield{u8} register number
		\cfield{u8[]} bytes to write
    \end{cmdreq} \\

	3 & \cname{READ\_REG}
	Read from a slave register. Writes the register number and issues a read transaction of the given length. Multiple registers can be read at once if the slave supports auto-increment.
	& \begin{cmdreq}
		\cfield{u16} slave address
		\cfield{u8} register number
		\cfield{u16} number of read bytes
    \end{cmdreq}
	\cjoin
    \begin{cmdresp}
		\cfield{u8[]} received bytes
    \end{cmdresp} \\

\end{cmdlist}


\section{USART Unit}

The USART unit provides access to one of the microcontroller's USART peripherals. All USART parameters can be configured to match the application's needs. The peripheral is capable of driving RS485 transceivers with the Driver Enable (DE) output for switching between reception and transmission.

The unit implements asynchronous reception and transmission using DMA and a circular buffer. Received data is sent to the host in asynchronous events when either half of the buffer is filled, or after a fixed timeout from the last received byte.

\subsection{USART Configuration}

\begin{inicode}
[USART:ser@6]
# Peripheral number (UARTx 1-4)
device=1
# Pin mappings (TX,RX,CK,CTS,RTS/DE)
#  USART1: (0) A9,A10,A8,A11,A12   (1) B6,B7,A8,A11,A12
#  USART2: (0) A2,A3,A4,A0,A1      (1) A14,A15,A4,A0,A1
#  USART3: (0) B10,B11,B12,B13,B14
#  USART4: (0) A0,A1,C12,B7,A15    (1) C10,C11,C12,B7,A15
remap=0

# Baud rate in bps (eg. 9600)
baud-rate=115200
# Parity type (NONE, ODD, EVEN)
parity=NONE
# Number of stop bits (0.5, 1, 1.5, 2)
stop-bits=1
# Bit order (LSB or MSB first)
first-bit=LSB
# Word width (7,8,9) - including parity bit if used
word-width=8
# Enabled lines (RX,TX,RXTX)
direction=RXTX
# Hardware flow control (NONE, RTS, CTS, FULL)
hw-flow-control=NONE

# Generate serial clock (Y,N)
clock-output=N
# Clock polarity: 0,1
cpol=0
# Clock phase: 0,1
cpha=0

# Generate RS485 Driver Enable signal (Y,N) - uses RTS pin
de-output=N
# DE active level: 0,1
de-polarity=1
# DE assert time (0-31)
de-assert-time=8
# DE clear time (0-31)
de-clear-time=8
\end{inicode}

\subsection{USART Events}


\begin{tabularx}{\textwidth}{p{\fldwcode}Xp{\fldwpld}}
	\toprule
	\textbf{Code} & \textbf{Meaning} & \textbf{Payload}  \\	
	\midrule	
	
	0 & \flname{DATA\_RECEIVED}
	 Data was received on the serial port.
	& \makecell[tl]{
		\fld{u8[]} received bytes
	} \\	
	\bottomrule
\end{tabularx}

\subsection{USART Commands}

\begin{tabularx}{\textwidth}{p{\fldwcode}Xp{\fldwpld}}
	\toprule
	\textbf{Code} & \textbf{Function} & \textbf{Payload}  \\	
	\midrule	
	
	0 & \flname{WRITE} 
	Add data to the transmit buffer. Sending is asynchronous, but the command may wait for free space in the DMA buffer.
	& \makecell[tl]{
		\fldreq
		\fld{u8[]} bytes to write	
	} \\	
	
	1 & \flname{WRITE\_SYNC}
	Add data to the transmit buffer and wait for the transmission to complete.
	& \makecell[tl]{
		\fldreq
		\fld{u8[]} bytes to write	
	} \\	

	\bottomrule
\end{tabularx}











\section{1-Wire Unit}

The 1-Wire unit implements the Dallas Semiconductor's 1-Wire protocol, most commonly used to interface smart thermometers (DS18x20). The protocol is explained in \cref{sec:theory-1wire}.

This unit implements the ROM Search algorithm that is used to find the ROM codes of all 1-Wire devices connected to the bus. The algorithm can find up to 32 devices in one run; more devices can be found by issuing the SEARCH\_CONTINUE command.

Devices are addressed using their ROM codes, unique 64-bit (8-byte) identifiers that work as addresses. When only one device is connected, the value 0 may be used instead and the addressing will be skipped. Its ROM code may be recovered using the READ\_ADDR command or by the search algorithm.

\subsection{1-Wire Configuration}

\begin{inicode}
[1WIRE:ow@7]
# Data pin
pin=A0
# Parasitic (bus-powered) mode
parasitic=N
\end{inicode}

\subsection{1-Wire Commands}

\begin{cmdlist}
	0 & \cname{CHECK\_PRESENCE}
	Test if there are any devices attached to the bus.
	& \begin{cmdresp}
		\cfield{u8} presence detected (0, 1)
	 \end{cmdresp} \\

	1 & \cname{SEARCH\_ADDR}
	Start the search algorithm.
	& \begin{cmdresp}
		\cfield{u8} should continue (0, 1)
		\cfield{u64[]} ROM codes
	 \end{cmdresp} \\

	2 & \cname{SEARCH\_ALARM}
	Start the search algorithm, finding only devices in an alarm state.
	& \begin{cmdresp}
		\cfield{u8} should continue (0, 1)
		\cfield{u64[]} ROM codes
	 \end{cmdresp} \\

	3 & \cname{SEARCH\_CONTINUE}
	Continue a previously started search
	& \begin{cmdresp}
		\cfield{u8} should continue (0, 1)
		\cfield{u64[]} ROM codes
	 \end{cmdresp} \\

	4 & \cname{READ\_ADDR}
	Read a device address (single device only)
	& \begin{cmdresp}
		\cfield{u64} ROM code
	 \end{cmdresp} \\

	10 & \cname{WRITE}
	Write bytes to a device.
	& \begin{cmdreq}
		\cfield{u64} ROM code
		\cfield{u8[]} bytes to write
	\end{cmdreq} \\

	11 & \cname{READ}
	Write a request and read response.
	&
	\begin{cmdreq}
		\cfield{u64} ROM code
		\cfield{u16} read length
		\cfield{u8} verify checksum (0, 1)
		\cfield{u8[]} request bytes
	\end{cmdreq}
	\cjoin
	\begin{cmdresp}
		\cfield{u8[]} read bytes
	\end{cmdresp}
	\\

	20 & \cname{POLL\_FOR\_1}
	Wait for a READY status, used by DS18x20.
	Not available in parasitic mode.
	Responds with SUCCESS after all devices are ready.
	& \\

\end{cmdlist}




\section{Frequency Capture Unit}

The frequency capture unit implements both the frequency measurement methods explained in section \ref{sec:theory-fcap}: direct and reciprocal. It can be operated in an on-demand or continuous measurement mode. The unit can be switched to two other modes: pulse counter, and the measurement of a single pulse.

\section{ADC Unit}

The analog/digital converter unit is one of the most complicated units implemented in the project. The unit can measure the voltage on an input pin, either as its immediate value, or averaged with exponential forgetting. Isochronous sampling is available as well: It's possible to capture a fixed-length block of data on demand, or as a response to a triggering condition on any of the enabled input pins. The ADC must continuously sample the inputs to make the averaging and level based triggering possible; As a consequence, a pre-trigger buffer is available that can be read together with the block of samples following a trigger. The ADC unit can also be switched to a continuous streaming mode.

It's possible to activate any number of the 16 analog inputs of the ADC peripheral simultaneously. The maximum continuous sampling frequency, which reaches 70\,ksps with one channel, lowers with an increasing number of enabled channels as the amount of data to transfer to the host increases.


\section{DAC Unit}

The digital/analog unit works with the two-channel \gls{DAC} hardware peripheral. It can be used in two modes: \gls{DC} output, and waveform generation.

The waveform mode implements direct digital synthesis (explained in section~\ref{sec:theory-dac-dds}) of a sine, rectangle, sawtooth or triangle wave. The generated frequency can be set with a sub-hertz precision up to the lower tens of kHz. The two outputs can use a different waveform shape, can be synchronized, and their phase offset and frequency are dynamically adjustable.

\subsection{DAC Configuration}

\begin{inicode}
[DAC:dac@13]
# Enabled channels (1:A4, 2:A5)
ch1_enable=Y
ch2_enable=Y
# Enable output buffer
ch1_buff=Y
ch2_buff=Y
# Superimposed noise type (NONE,WHITE,TRIANGLE) and nbr. of bits (1-12)
ch1_noise=NONE
ch1_noise-level=3
ch2_noise=NONE
ch2_noise-level=3
\end{inicode}

\subsection{DAC Commands}

Channels are specified in all commands as a bit map:

\begin{itemize}[nosep]
	\item 0x01 - channel 1
	\item 0x02 - channel 2
	\item 0x03 - both channels affected at once
\end{itemize}

\begin{cmdlist}
	0 & \cname{WAVE\_DC}
	Set a \gls{DC} level, disable \gls{DDS} for the channel
	& \begin{cmdreq}
		\cfield{u8} channels
		\cfield{u16} level (0--4095)
	\end{cmdreq} \\

	1 & \cname{WAVE\_SINE}
	Start a sine waveform
	& \begin{cmdreq}
		\cfield{u8} channels
	\end{cmdreq} \\

	2 & \cname{WAVE\_TRIANGLE}
	Start a symmetrical triangle waveform
	& \begin{cmdreq}
		\cfield{u8} channels
	\end{cmdreq} \\

	3 & \cname{WAVE\_SAWTOOTH\_UP}
	Start a rising sawtooth waveform
	& \begin{cmdreq}
		\cfield{u8} channels
	\end{cmdreq} \\

	4 & \cname{WAVE\_SAWTOOTH\_DOWN}
	Start a dalling sawtooth waveform
	& \begin{cmdreq}
		\cfield{u8} channels
	\end{cmdreq} \\

	5 & \cname{WAVE\_RECTANGLE}
	Start a rectangle waveform
	& \begin{cmdreq}
		\cfield{u8} channels
		\cfield{u16} on-time (0--8191)
		\cfield{u16} high level (0--4095)
		\cfield{u16} low level (0--4095)
	\end{cmdreq} \\

	10 & \cname{SYNC}
	Synchronize the two channels. The phase accumulator is reset to zero.
	& \\

	20 & \cname{SET\_FREQUENCY}
	Set the channel frequency
	& \begin{cmdreq}
		\cfield{u8} channels
		\cfield{float32} frequency
	\end{cmdreq} \\

	21 & \cname{SET\_PHASE}
	Set a channel's phase. It is recommended to only set the phase of one channel, leaving the other at 0°.
	& \begin{cmdreq}
		\cfield{u8} channels
		\cfield{u16} phase (0--8191)
	\end{cmdreq} \\

	22 & \cname{SET\_DITHER}
	Control the dithering function of the \gls{DAC} block. A high noise amplitude can cause an overflow to the other end of the output range due to a bug in the \gls{DAC} peripheral. Use value 255 to leave the parameter unchanged.

	& \begin{cmdreq}
		\cfield{u8} channels
		\cfield{u8} noise type (0-none, 1-white, 2-triangle)
		\cfield{u8} number of noise bits (1--12)
	\end{cmdreq} \\
\end{cmdlist}




\section{PWM Unit}

The \gls{PWM} unit uses a timer/counter to generate a \gls{PWM} signal. There are four outputs with a common frequency and phase, but independent duty cycles. Each channel can be individually enabled or disabled.

This unit is intended for applications like light dimming, heater regulation, or the control of H-bridges.

\todo[inline]{diagram, also show what is duty cycle}

\subsection{PWM Configuration}

\begin{inicode}
[PWMDIM:pwm@12]
# Default pulse frequency (Hz)
frequency=1000
# Pin mapping - 0=disabled
# Channel1 - 1:PA6, 2:PB4, 3:PC6
ch1_pin=1
# Channel2 - 1:PA7, 2:PB5, 3:PC7
ch2_pin=0
# Channel3 - 1:PB0, 2:PC8
ch3_pin=0
# Channel4 - 1:PB1, 2:PC9
ch4_pin=0
\end{inicode}

\subsection{PWM Commands}

\begin{cmdlist}
    0 & \cname{SET\_FREQUENCY}
    Set the PWM frequency
    & \begin{cmdreq}
        \cfield{u32} frequency in Hz
    \end{cmdreq} \\

    1 & \cname{SET\_DUTY}
    Set the duty cycle of one or more channels
    & \begin{cmdreq}
        \item Repeat 1--4 times:
        \begin{pldlist}
            \cfield{u8} channel number 0--3
            \cfield{u16} duty cycle 0--1000
        \end{pldlist}
    \end{cmdreq} \\

    2 & \cname{STOP}
    Stop the hardware timer. Outputs enter low level.
    & \\

    3 & \cname{START}
    Start the hardware timer.
    & \\
\end{cmdlist}




\section{Touch Sense Unit}

The touch sensing unit provides an access to the \gls{TSC} peripheral. Its function is explained in section \ref{sec:theory-touch}. The unit configures the \gls{TSC} and reads the output values of each enabled touch pad.


\subsection{Touch Sense Configuration}

\begin{inicode}
[TOUCH:touch@11]
# Pulse generator clock prescaller (1,2,4,...,128)
pg-clock-prediv=32
# Sense pad charging time (1-16)
charge-time=2
# Charge transfer time (1-16)
drain-time=2
# Measurement timeout (1-7)
sense-timeout=7

# Spread spectrum max deviation (0-128,0=off)
ss-deviation=0
# Spreading clock prescaller (1,2)
ss-clock-prediv=1

# Optimize for interlaced pads (individual sampling with others floating)
interlaced-pads=N

# Button mode debounce (ms) and release hysteresis (lsb)
btn-debounce=20
btn-hysteresis=10

# Each used group must have 1 sampling capacitor and 1-3 channels.
# Channels are numbered 1,2,3,4

# Group1 - 1:A0, 2:A1, 3:A2, 4:A3
g1_cap=
g1_ch=
# Group2 - 1:A4, 2:A5, 3:A6, 4:A7
g2_cap=
g2_ch=
# ...
\end{inicode}


\subsection{Touch Sense Events}

\begin{cmdlist}
    0 & \cname{BUTTON\_CHANGE}
    The binary state of some of the capacitive pads with button mode enabled changed.
    & \begin{cmdpld}
        \cfield{u32} binary state of all channels
        \cfield{u32} changed / trigger-generating channels
    \end{cmdpld} \\
\end{cmdlist}

%\newpage
\subsection{Touch Sense Commands}

\begin{cmdlist}
    0 & \cname{READ}
    Read the raw touch pad values (lower indicates higher capacitance). Values are ordered by group and channel.
    & \begin{cmdreq}
        \cfield{u16[]} raw values
    \end{cmdreq} \\

    1 & \cname{SET\_BIN\_THR}
    Set the button mode thresholds for all channels. Value 0 disables the button mode for a channel.
    & \begin{cmdreq}
        \cfield{u16[]} thresholds
    \end{cmdreq} \\

    2 & \cname{DISABLE\_ALL\_REPORTS}
    Set thresholds to 0, disabling the button mode for all pads.
    & \\

    3 & \cname{SET\_DEBOUNCE\_TIME}
    Set the button mode debounce time (used for all pads with button mode enabled).
    & \begin{cmdreq}
        \cfield{u16} debounce time (ms)
    \end{cmdreq} \\

    4 & \cname{SET\_HYSTERESIS}
    Set the button mode hysteresis.
    & \begin{cmdreq}
        \cfield{u16} hystheresis
    \end{cmdreq} \\

\end{cmdlist}











