% Abstrakt

\begin{abstract-english}
This thesis documents the development of a general-purpose software and hardware platform for the interfacing of low-level hardware from high-level programming languages and applications run on the PC, using USB and also wirelessly.

The requirements of common engineering tasks and problems occurring in the university environment were evaluated to design an extensible, reconfigurable hardware module that would make a practical, versatile, and low-cost tool that in some cases eliminates the need for professional measurement and testing equipment.

Two hardware prototypes were designed and realized, accompanied by control libraries for programming languages C and Python. The Python library additionally integrates with MATLAB scripts. The devices provide access to hardware buses (\IIC, SPI, USART, 1-Wire) and microcontroller peripherals (ADC, DAC), implement frequency measurement and other useful features. The device is parametrised by a configuration file on a virtual disk accessible through USB, or written programmatically.
\end{abstract-english}

\begin{abstract-czech}
\begin{otherlanguage}{czech}
Tato práce popisuje vývoj univerzální softwarové a~hardwarové platformy pro přístup k~hardwarovým sběrnicím a~elektrickým obvodům z~prostředí vysokoúrovňových programovacích jazyků a~aplikací běžících na PC, a~to za využití USB a~také bezdrátově.

Byly vyhodnoceny požadavky typických problémů, vyskytujících se v~praxi při práci s~vestavěnými systémy a~ve výuce, pro návrh snadno rozšiřitelného a~přenastavitleného hardwarového modulu který bude praktickým, pohodlným a~dostupným nástrojem který navíc v~některých případech může nahradit profesionální laboratorní přístroje.

Bylo navrženo několik prototypů hardwarových modulů, spolu s~obslužnými knihovnami v~jazycích C a~Python; k~modulu lze také přistupovat z~prostředí MATLAB. Přístroj umožňuje přístup k~většině běžných hardwarových sběrnic a~umožňuje také např. měřit frekvenci a~vzorkovat či generovat analogové signály.
\end{otherlanguage}
\end{abstract-czech}
