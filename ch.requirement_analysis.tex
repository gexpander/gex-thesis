\chapter{Requirement Analysis}

We'll now investigate some situations where GEX could be used, to establish its requirements and desired features.

\section{Desired Features}

\subsection{Interfacing Intelligent Modules}\label{sec:uses_digital_ifaces}

When adding a new digital sensor or a module to a hardware project, we want to test it first, learn how to properly communicate with it, and confirm its performance. Based on this evaluation we decide whether the module matches our expectations and learn how to properly connect it, which is needed for a successful \gls{PCB} layout.

In experimental setups, this may be the only thing we need. Data can readily be collected after just connecting the module to a \gls{PC}, same as commanding motor controllers or other intelligent devices.

A couple of well known hardware buses have established themselves as the standard ways to interface digital sensors and modules: \gls{SPI}, \gls{I2C} and \gls{USART} (\gls{UART} in asynchronous mode) are some of the most common ones, often accompanied by a few extra \gls{GPIO} lines for features such as Reset, Chip Enable, or Interrupt. There are exceptions where silicon vendors have developed proprietary communication protocols that continue to be used either for historical reasons, or because of their specific advantages. An example is the Dallas Semiconductor 1-Wire bus used in digital thermometers.

Moving to industrial and automotive environments, we encounter various fieldbuses, Ethernet, \gls{CAN}, current loop, \gls{HART}, \gls{LIN}, \gls{DALI}, RS-485 (e.g., for Modbus), \gls{mbus}, PLC-BUS, and others. Those typically use transceiver \glspl{IC} and other circuitry, such as \glspl{TVS}, signal filters, or galvanic isolation. They could be supported using add-on boards and additional firmware modules handling the protocol. For simplicity and to meet time constraints, the development of those boards and modules will be left for future expansions of the project.

\subsection{Analog Signal Acquisition}

Sometimes it is necessary to use a traditional analog sensor, capture a transient waveform, or to just measure voltage. GEX is meant to focus on digital interfaces, however giving it this capability makes it much more versatile. Nearly all microcontrollers include an \gls{ADC} which we can use to measure input voltages and, paired with a timer, to records signals varying in time.

Certain tasks, such as capturing transient effects on a thermocouple when inserted into a flame (an example from developing fire-proof materials) demand level triggering similar to that of oscilloscopes. The converter continuously measures the input voltage and a timed capture starts only after a set threshold is exceeded. This can be accompanied by a pre-trigger feature where the timed capture is continuously running and the last sample is always compared with the threshold, recording a portion of the historic records together with the following samples.

\subsection{Analog Signal Output}

An analog signal can not only be measured, but it is often necessary to also generate it. This could serve as an excitation signal for an experiment, for instance to measure the characteristic curves of a diode or a transistor. Conveniently, we can at the same time use GEX's analog input to record the output.

Generating an analog signal is possible using a \gls{PWM} or by a dedicated digital-analog converter included in many microcontrollers. Higher frequencies or resolution can be achieved with a dedicated external \gls{IC}.

\subsection{Logic Level Input and Output}

We have covered some more advanced features, but skipped the simplest feature: direct access to \gls{GPIO} pins. Considering the latencies of \gls{USB} and the \gls{PC}'s \gls{OS}, this cannot be used reliably for ``bit banging''; however, we can still accomplish a lot with just changing logic levels---e.g., to control character \glspl{LCD}, or emulate some interfaces that include a clock line, like \gls{SPI}. As mentioned in \cref{sec:uses_digital_ifaces}, many digital sensors and modules use plain \glspl{GPIO} in addition to the communication bus for out-of-band signaling or features like chip selection or reset.

\subsection{Pulse Generation and Measurement}

Some sensors have a variable frequency or a \gls{PWM} output. To capture those signals and convert them to a more useful digital value, we can use the external input functions of a timer/counter in the microcontroller. Those timers have many possible configurations and can also be used for pulse counting or waveform generation.

\section{Connection to the Host Computer}

\subsection{Communication Interface}

\gls{USB} shall be the primary way of connecting the module to a host \gls{PC}. Thanks to \gls{USB}'s flexibility, it can present itself as any kind of device or even multiple devices at once.

The most straightforward method of interfacing the board is by passing binary messages in a fashion similar to \gls{UART}. We'll need a duplex connection to enable command confirmations, query-type commands and asynchronous event reporting. This is possible either using a ``Virtual COM port'' driver, or through raw access to the corresponding \gls{USB} endpoints. Using raw access avoids potential problems with the \gls{OS}'s driver interfering or not recognizing the device correctly; on the other hand, having GEX appear as a serial port makes it easier to integrate into existing platforms that have good serial port support (such as National Instruments LabWindows CVI or MATLAB).

A connection using a hardware \gls{UART} is also planned, as a fallback for boards without an USB connector or for platforms with no \gls{USB} connectivity. A wireless connection to the host PC should also be possible and work transparently in a similar way to the \gls{USB} or \gls{UART} connection.

\subsection{Configuration Files}

The module must be easily reconfigurable. Given the settings are almost always going to be tied to the connected external hardware, it would be practical to have an option to store them permanently in the microcontroller's non-volatile memory.

We can load those settings into GEX using the serial interface, which also makes it possible to reconfigure it remotely when the wireless connection is used. With USB, we can additionally make the board appear as a mass storage device and expose the configuration as text files. This approach, inspired by \mbed's mechanism for flashing firmware images to development kits, avoids the need to create a configuration \gls{GUI}, instead using the built-in applications of the \gls{PC} \gls{OS} to view and edit files. Besides the configuration files, we can expose additional information, such as a README file with instructions, or a pin-out reference, as separate files on the virtual disk.

\section{An Overview of Planned Features}

Summarizing the preceding discussion, we obtain the following list of features to implement in the GEX firmware:

\begin{itemize}
	\item \textbf{Hardware interfacing functions}
		\begin{itemize}
			\item I/O pin direct access (read, write), pin change interrupt
			\item Analog input: voltage measurement, sampled capture
			\item Analog output: static level, waveform generation
			\item Frequency, duty cycle, pulse length measurement
			\item Single pulse and \gls{PWM} generation
			\item \gls{SPI}, \gls{I2C}, \gls{UART}/\gls{USART}, 1-Wire
%			\item NeoPixel (addressable \gls{LED} strips)
		\end{itemize}
	\pagebreak[0]
	\item \textbf{Communication with the host computer}
		\begin{itemize}
			\item \gls{USB} connection as virtual serial port or direct endpoint access
			\item Connection using plain \gls{UART}
			\item Wireless attachment
		\end{itemize}
	\item \textbf{Configuration}
		\begin{itemize}
			\item Fully reconfigurable, temporarily or permanently
			\item Settings stored in INI files
			\item File access through the communication interface or using a virtual mass storage
		\end{itemize}
\end{itemize}

\section{Microcontroller Selection}

As discussed in \cref{sec:expected_outcome}, this project will be based on microcontrollers from the STM32 family. The STM32F072 model was selected for the initial hardware and firmware design due to its low cost, advanced peripherals, and the availability of development boards. The firmware can be ported to other \glspl{MCU} later (e.g., to STM32L072, STM32F103 or STM32F303).

The STM32F072 is an \armcm device with 128\,KiB of flash memory, 16\,KiB of \gls{RAM} and running at 48\,MHz. It is equipped with a \gls{USB} Full Speed peripheral block, a 12-bit \gls{ADC} and \gls{DAC}, a number of general-purpose timers/counters, SPI, I$^2$C, and USART peripherals, among others. It supports crystal-less \gls{USB}, using the USB SOF packet for synchronization of the internal 48\,MHz RC oscillator; naturally, a real crystal resonator will provide better timing accuracy.

To effectively utilize the time available for this work, only the STM32F072 firmware will be developed while making sure the planned expansion is as straightforward as possible.

\section{Form Factor Considerations} \label{sec:formfactors}

While the GEX firmware can be used with existing evaluation boards from ST Microelectronics (\cref{fig:discovery}), we wish to design and realize a few custom hardware prototypes that will be smaller and more convenient to use.

Three possible form factors are drawn in \cref{fig:ff_sketches}. The use of a common connector layout and pin assignments, here Arduino and Raspberry Pi, makes it possible to reuse add-on boards from those platforms. When we copy the physical form factor of another product, in this example the Raspberry Pi Zero, we can further take advantage of existing enclosures designed for it.

\begin{figure}[h]
	\centering
	\includegraphics[width=0.7\textwidth] {img/disco072.jpg}
	\caption[A Discovery board with STM32F072]{\label{fig:discovery}A Discovery development board with the STM32F072 microcontroller}
\end{figure}

\begin{figure}[h]
	\centering
	\includegraphics[width=\textwidth] {img/gex-ff-sketches.png}
	\caption[Form factor sketches]{\label{fig:ff_sketches}A sketch of three possible form factors for a GEX hardware realization}
\end{figure}









